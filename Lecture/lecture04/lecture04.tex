
\documentclass[xcolor={dvipsnames}]{beamer}
\usepackage{amsmath,amsfonts,amssymb,pxfonts,eulervm,xspace}
\usepackage{graphicx}
 \usepackage{multimedia}
\usepackage{media9}

\graphicspath{{./figures/}}
\usetheme{ccnycrest}


\newenvironment{changemargin}[2]{%
\begin{list}{}{%
\setlength{\topsep}{0pt}%
\setlength{\leftmargin}{#1}%
\setlength{\rightmargin}{#2}%
\setlength{\listparindent}{\parindent}%
\setlength{\itemindent}{\parindent}%
\setlength{\parsep}{\parskip}%
}%
\item[]}{\end{list}}

\begin{document}

\title{ CS102: Declarations, Initialization, and Assignment }
\author{Hannah Aizenman}
\date{haizenm00@ccny.cuny.edu}


\begin{frame}
	\titlepage
\end{frame}


\begin{frame}{Memory}

\begin{block}
	\begin{figure}
		\includegraphics[width=1\textwidth]{memory}
	\end{figure}
\end{block}

%\begin{block}
%	\begin{description}
%		\item[stack] - memory set aside for an execution thread (program)
%		\item[heap] - memory set aside for allocating on the fly
%	\end{description}
%\end{block}

\end{frame}
%
%
%\begin{frame}{Declaration Statement}
%
%\end{frame}
%
%\begin{frame}{Declaration Statement}
%	\begin{block}{How they work:}
%		\begin{itemize}
%			\item identifies the variable \textbf{name}
%			\item specifies the variable \textbf{type}
%		\end{itemize}
%	\end{block}
%	\pause
%	\begin{block}{How they're written:}
%		\begin{center}
%			\textcolor{DarkOrchid}{\textbf{type}} \textit{name};\\
%			\pause
%			\textcolor{DarkOrchid}{\textbf{type}} \textit{name1}, \textit{name2}, \textit{name3}, ...;
%		\end{center}
%	\end{block}
%\end{frame}


\end{document}

