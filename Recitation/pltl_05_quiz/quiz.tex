\documentclass[addpoints,12pt]{exam}
\usepackage[margin=1in]{geometry}
\usepackage{listings}
\usepackage{tabularx}
\newcounter{matchleft}
\newcounter{matchright}

\newenvironment{matchtabular}{%
  \setcounter{matchleft}{0}%
  \setcounter{matchright}{0}%
  \tabularx{\textwidth}{%
    >{\leavevmode\hbox to 1.5em{\stepcounter{matchleft}\arabic{matchleft}.}}X%
    >{\leavevmode\hbox to 1.5em{\stepcounter{matchright}\alph{matchright})}}X%
    }%
}{\endtabularx}

\begin{document}
\header{CS102}{Quiz 4}{}

\begin{center}
\fbox{\fbox{\parbox{5.5in}{\centering
Answer the questions in the spaces provided on the
question sheets. If you run out of room for an answer,
continue on the back of the page.\\
\textbf{Show all work}. \\
Credit \textbf{will not} be given if work is not shown.}}}
\end{center}
\vspace{0.1in}
\makebox[\textwidth]{Name and section:\enspace\hrulefill}
\begin{center}
\gradetable[h][questions]
\end{center}

\begin{questions}
\question [5]
Given the following code:
\begin{lstlisting}{language=c++}
#include <iostream>
using namespace std;

int main(){
int y;
for (int x=1; x<=5; x++){
    for(int z=2; z<=6; z++){
        y=(x*z)/(1.0*(x-z));
        cout<<"f("<<x<<","<<z<<")=";
        if (x==z){
            cout<<"NaN"<<endl;
        }else{
            cout<<y<<endl;
        }
    }
}
return 0;
}
\end{lstlisting}
Fill out a table describing how the program behaves and include the following columns:  loop \# (iteration), loop \# (iteration), x, z, y, and output.

\break
\question [5]
Given the following code:
\begin{lstlisting}{c++}
#include <iostream>
#include <cstdlib>
#include <ctime>

using namespace std;
int main(){
int guess;
const int MAXNUM = 100;

/* initialize random seed: */
srand (time(NULL));

/* generate random number between 0 and MAXNUM */
int rnumber = rand() % MAXNUM;

do{
    cout<<"What's your guess? ";
    cin>>guess;
    if (guess>rnumber){
        cout<<"Too high, try again."<<endl;
    }else if(guess<rnumber){
        cout<<"Too low, try again."<<endl;
    }else{
        cout<<"Correct!"<<endl;
    }
}while(guess!=rnumber);

return 0;
}
\end{lstlisting}
Given that  random number is 28 and the guesses are: 50, 25, 37, 29, 27, 28\\
Fill out a table describing how the program behaves and include the following columns: loop \# (iteration), guess, output.

\end{questions}



\end{document}