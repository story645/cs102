\documentclass{article}
\usepackage{graphicx}
\usepackage[margin=1in]{geometry}
\usepackage{listings}
\usepackage{amsmath}

\begin{document}

\title{CS102: Week 10}

\maketitle
\section*{Character handling functions}
\begin{tabular}{|l|l|}
\hline
isalnum & Check if character is alphanumeric (function )\\ 
 \hline 
isalpha & Check if character is alphabetic (function )\\ 
 \hline 
isblank & Check if character is blank (function )\\ 
 \hline 
iscntrl & Check if character is a control character (function )\\ 
 \hline 
isdigit & Check if character is decimal digit (function )\\ 
 \hline 
isgraph & Check if character has graphical representation (function )\\ 
 \hline 
islower & Check if character is lowercase letter (function )\\ 
 \hline 
isprint & Check if character is printable (function )\\ 
 \hline 
ispunct & Check if character is a punctuation character (function )\\ 
 \hline 
isspace & Check if character is a white-space (function )\\ 
 \hline 
isupper & Check if character is uppercase letter (function )\\ 
 \hline 
isxdigit & Check if character is hexadecimal digit (function )\\ 
 \hline 
tolower & Convert uppercase letter to lowercase (function )\\ 
 \hline 
toupper & Convert lowercase letter to uppercase (function ) \\
\hline
\end{tabular}
\section*{Sentences}
Write a program that asks the user to enter in a sentence (for example "the quick brown fox") and:
\begin{enumerate}
	\item counts the number of words in the sentence
	\item capitalizes every other word
\end{enumerate}

\begin{lstlisting}{c++}
//pulls in individual words-stops at space
string input;    
cin>>input;
//pulls in all words until newline/enter in input
//cin is the stream the words wre coming from
getline(cin, input);
\end{lstlisting}
 getline(cin, var) is for sentences/lines of words.

\section*{Counting Letters}
Write a program that asks the user to enter in a sentence (for example "the quick brown fox") and:
\begin{enumerate}
	\item counts the number of vowels
	\item counts the number of consonants
\end{enumerate}

\section*{Exception handling}
Write a function to solve for the roots of a quadratic equation. Your function should return the first positive root and throw exceptions when 
\begin{enumerate}
	\item division by 0 would happen
	\item when the roots are imaginary
\end{enumerate}
 Use the provided prototype:
\begin{lstlisting}
	double quadratic(int a, int b, int c);
\end{lstlisting}
Write a main program that tests this function. 


\end{document}