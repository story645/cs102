\documentclass[addpoints,12pt]{exam}
\usepackage[margin=1in]{geometry}
\usepackage{listings}
\usepackage{multicol}
\usepackage{tabularx}
\newcounter{matchleft}
\newcounter{matchright}

\newenvironment{matchtabular}{%
  \setcounter{matchleft}{0}%
  \setcounter{matchright}{0}%
  \tabularx{\textwidth}{%
    >{\leavevmode\hbox to 1.5em{\stepcounter{matchleft}\arabic{matchleft}.}}X%
    >{\leavevmode\hbox to 1.5em{\stepcounter{matchright}\alph{matchright})}}X%
    }%
}{\endtabularx}

\begin{document}
\header{CS102}{Quiz 10}{}

\begin{center}
\fbox{\fbox{\parbox{5.5in}{\centering
Answer the questions in the spaces provided on the
question sheets. If you run out of room for an answer,
continue on the back of the page.\\
\textbf{Show all work}. \\
Credit \textbf{will not} be given if work is not shown.}}}
\end{center}
\vspace{0.1in}
\makebox[\textwidth]{Name and section:\enspace\hrulefill}
\begin{center}
\gradetable[h][questions]
\end{center}

\begin{questions}
\question [6]
Given the code on the following page, what is the output when the input is:
\begin{parts}
	\part[2]
		Hello!
	\part[2]
		The quick brown fox jumped over the lazy river.
	\part[2]
		Homework was due yesterday, not tomorrow.
\end{parts}

\question[1]
Why is getline used instead of cin?
\question[1]
What is flip used for?
\question[1]
What is the purpose of string output=input?
\question[1]
What does count measure?

\end{questions}
\pagebreak
\begin{lstlisting}{c++}
#include <iostream>
#include <cstdlib>
#include <string> 
using namespace std;
string transform(string input);

int main(){
	string input, output;
	cout<<"Enter sentence: ";
	getline(cin, input);
	try{
		output = transform(input);
	}catch(string mc){
		cout<<"count: "<<mc<<endl;
	}
	cout<<"output: "<<output<<endl; 
	return 0;
}
string transform(string input){
	int count = 0; 
	bool flip = false;
	string output = input;
	for(int i=0; i<input.length(); i++){ 
		if(isblank(input.at(i))){ 
			count++; 
			flip = !flip;
		}
		if (flip){
			output[i] = toupper(input.at(i));
		}
		if(ispunct(input.at(i))){
			break; 
		}
	}
	if (count<1){
		string error = "count<1";
		throw error;
	}
	
	return output;
}
\end{lstlisting}

\end{document}