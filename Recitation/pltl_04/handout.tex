\documentclass{article}
\usepackage{graphicx}
\usepackage[margin=1in]{geometry}
\usepackage{listings}
\usepackage{amsmath}

\begin{document}

\title{CS102: Week 4}

\maketitle

\section*{Fizz Buzz 2}
Write, run, and test a C++ program that for all numbers between and 0 and 100 (inclusive) prints out the number and
\begin{itemize}
	\item print \textbf{Fizz} for multiples of 3
	\item print \textbf{Buzz} for multiples of 5
	\item print \textbf{FizzBuzz} for multiples of 3 and 5
\end{itemize}


\section*{Exponents: $a^{b}$}
\begin{enumerate}
	\item Write, run, and test a C++ program to find the value of $2^{n}$ by using a for loop, where n is an integer value the user
enters.
	\item Write, run, and test a C++ program to find the value of $a^{b}$  by using a for loop, where a and b are integer values that the user enters. Use the pow function from the math library to verfiy your solution.
\end{enumerate}

\section*{Approximating Euler's constant}
Euler's constant E can be approximated as a series of terms using the Taylor series:\\

\begin{enumerate}

\item Write, run, and test a C++ program to compute the approximation of e using N terms:\\
$e = \displaystyle \sum^{\infty  }_{n=0}\frac{1}{n!} = 1 + \frac{1}{1!} + \frac{1}{2!} + \frac{1}{3!} .....$ for all x
\item Write, run, and test a C++ program to compute the approximation of $e^{x}$ using N terms:\\
$e^{x}=\displaystyle \sum^{\infty}_{n=0}\frac{x^{n}}{n!} = 1 + \frac{x}{1!} + \frac{x^{2}}{2!} + \frac{x^{n}}{3!}.....$ for all x
\end{enumerate}
\end{document}