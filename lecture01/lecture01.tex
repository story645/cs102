
\documentclass[xcolor={dvipsnames}]{beamer}
\usepackage{amsmath,amsfonts,amssymb,pxfonts,eulervm,xspace}
\usepackage{graphicx}
 \usepackage{multimedia}

\usetheme{ccnycrest}
\begin{document}

\title{ CS102: Lecture 01}
\author{Hannah Aizenman}
\date{haizenm00@ccny.cuny.edu}


\begin{frame}
	\titlepage
\end{frame}

\begin{frame}{Grading}
	\begin{itemize}
		\item \%10 - Recitation Attendance/Quiz
		\item \%30 - Homework (\%20 code, \% 10 Questions)
		\item \%25 - Midterm
 		\item \%35 - Final
	\end{itemize}
\end{frame}

\begin{frame}{Plagerism}
	Plagerizing homework:
	\begin{itemize}
		\item 1st - F on the assignment
		\item 2nd - F for all homeworks
		\item 3rd - F for the class
	\end{itemize}
	Cheating on exams:
	\begin{itemize}
		\item midterm - F on the midterm
		\item final - F on the final
	\end{itemize}
\end{frame}

\begin{frame}{Contact Info}
\begin{itemize}
	\item Hannah Aizenman
	\item haizenm00@ccny.cuny.edu
	\item office hours: NAC 7/311
	\item office hours: 3:00-5:00 Th and by appointment
\end{itemize}
\end{frame}

\begin{frame}{Required Materials}
\begin{itemize}
	\item Textbook: Gary J. Bronson, C++ for Engineers and Scientists (any edition)
	\item Text Editor: Notepad, Notepad++ or any other plaintext editor
	\item Compiler: g++ (install on Windows using MinGW or Cygwin, on Mac using xcode)
\end{itemize}
\end{frame}

\begin{frame}{History of CS: 1805}
	\begin{columns}
	 \begin{column}{.49\textwidth}
	\begin{itemize}
	\item Joseph-Marie Jacquard
	\item punch card programs
	\item programmable loom
	\end{itemize}
 	\end{column}
	 \begin{column}{.49\textwidth}
  		\begin{figure}
 		\includegraphics[scale=0.5]{Jacquard_Joseph_Marie_woven_silk}
		\end{figure}
	\end{column}
\end{columns}
\end{frame}

\begin{frame}{History of CS: 1837}
	\begin{columns}
	 \begin{column}{.49\textwidth}
		\begin{center}Charles Babbage\end{center}
		\begin{figure}
		\includegraphics[scale=0.5]{Charles_Babbage}
		\caption{Invented first computer}
		\end{figure}
 	\end{column}
	 \begin{column}{.49\textwidth}
		\begin{center}Ada Lovelace\end{center}
  		\begin{figure}
 		\includegraphics[scale=.60]{Ada_Lovelace_portrait}
		\caption{Wrote first program}
		\end{figure}
	\end{column}
\end{columns}
\end{frame}
\begin{frame}{Difference Engine-Polynomial Calculator}
	\begin{center}
	\movie{\includegraphics[width=0.8\textwidth]{Babbage_Difference_Engine}}{difference_engine.mp4}
	\end{center}
\end{frame}

\begin{frame}{History of CS: 1890}
	\begin{columns}
	 \begin{column}{.49\textwidth}
			\begin{center}Herman Hollerith\end{center}
  			\begin{figure}
 				\includegraphics[scale=.50]{Hollerith}
			\end{figure}
 	\end{column}
	 \begin{column}{.49\textwidth}
		\begin{center}Punchcard Tabulator\end{center}
  		\begin{figure}
 		\includegraphics{Card_puncher_NARA_513295}
		\caption{Hollerith card puncher used by the United States Census Bureau}
		\end{figure}
	\end{column}
\end{columns}
\end{frame}

\begin{frame}{The Tabulator}
	\begin{center}
	\href{http://channel.nationalgeographic.com/channel/the-link/videos/the-tabulator/}{\beamergotobutton{NatGeo}}
	\end{center}
\end{frame}

\begin{frame}{History of CS: WW11}
	\begin{columns}
	 \begin{column}{.49\textwidth}
			\begin{itemize}
				\item Alan Turing
				\item father of theoretical computer science and AI
				\item also code breaker
				\item turing machine reduces computation to lowest level
			\end{itemize}
 	\end{column}
	 \begin{column}{.49\textwidth}
  		\begin{figure}
 		\includegraphics{Alan_Turing_photo}
		\end{figure}
	\end{column}
\end{columns}
\end{frame}

\begin{frame}{Turing Machine}
	\begin{center}
	\movie{\includegraphics[width=0.8\textwidth]{lego_tm}}{turing_machine.mp4}
	\end{center}
\end{frame}


\begin{frame}{First Homework}
Fill out survey on blackboard, due 11:59PM Wednesday
\end{frame}
\end{document}